\documentclass[../main.tex]{subfiles}
\begin{document}
%% Don't try to compile this file, it is not a complete tex file
% \usepackage{bibentry}
\linenumbers            % remove before final print
% \listoftodos[Notes and To-Do:]

%% https://tex.stackexchange.com/questions/217517/handle-bib-in-subfiles
%% accessed 2017-12-17
\providecommand{\main}{..}
%**************************************************************
%-----------------> CONTENT STARTS HERE <----------------------
%**************************************************************
\chapter{LaTeX examples}
% \appendix{LaTeX examples}

This is a section to test formatting and figure placement in LaTeX.
%_-_-_-_-_-_-_-_-_-_-_-_-_-_-_-_-_
\section{Citations}
In-line paragraph citation:\par
As much as 1600-m of uplift has occurred along the Eastern Sierran range front \citep{Rinehart1964} over the past 5 Ma.

In-line year paragraph citation:
Early mapping of the region by \citet{Rinehart1964} and later by \citet{Bailey1974} suggested the HCF once extended into the area now occupied by the LVC.
\citep{Ponce2013}

This is an empty citation.  It is intended to be used as a placeholder until a proper reference can be identified \citep{addReference}
\begin{verbatim}
    \citep{addReference}
\end{verbatim}

%_-_-_-_-_-_-_-_-_-_-_-_-_-_-_-_-_
\section{Lists and bullets}
Basic list (aka "itemize")\par
	\begin{itemize}
		\item Stuff
		\item More stuff
		\item Even more stuff
	\end{itemize}

%_-_-_-_-_-_-_-_-_-_-_-_-_-_-_-_-_
\section{Figures}
Although partially filled in by Bishop Tuff and post-eruptive sediments, the Long Valley Caldera is a prominent feature on the landscape today as a 17 km by 32 km elongate depression (Figure \ref{fig:example1}).
\begin{figure}[ht]
	\centering
	\includegraphics[width=9cm]{figures/overview}
	\caption{Overview map of the Long Valley area.  Modified from \citet{Taylor1980}.}
	\label{fig:example1}
\end{figure}
\begin{verbatim}
    Figure \ref{fig:overview}
    \begin{figure}[ht]
    	\centering
    	\includegraphics[width=9cm]{figures/overview}
    	\caption{caption_text_here \citet{Taylor1980}.}
    	\label{fig:example1}
    \end{figure}
\end{verbatim}


\subsection{Code snippet as figure}
We can also create a figure that contains Python code (Figure \ref{fig:example2}).

	\begin{figure}[ht]
		\centering
        % This is an example of inserting code in the thesis.  This example is Python2

\begin{lstlisting}
def rotate_xy(xPoint,yPoint,x0,y0,angle):
	import math
	# angle in radians

	xDiff = xPoint - x0
	yDiff = yPoint - y0
	xNew = x0 + xDiff * math.cos(angle) - yDiff * math.sin(angle)
	yNew = y0 + yDiff * math.cos(angle) + xDiff * math.sin(angle)
    return xNew,yNew
\end{lstlisting}
    	\caption{A Python function to rotate a point \textit{(xPoint, yPoint)} about an origin \textit{(x0,y0)} by \textit{(angle)} degrees.}
    	\label{fig:example2}
	\end{figure}

\begin{verbatim}
	\begin{figure}[ht]
    	\centering
        \begin{lstlisting}
		    <code goes here>
		\end{lstlisting}
    	\caption{caption_text}
    	\label{fig:example2}
	\end{figure}
\end{verbatim}

%_-_-_-_-_-_-_-_-_-_-_-_-_-_-_-_-_
\section{Tables}
\begin{table}[ht]
    \centering
    \begin{tabular}{ |c|c|c| } 
         \hline
         cell1 & cell2 & cell3 \\ 
         cell4 & cell5 & cell6 \\ 
         cell7 & cell8 & cell9 \\ 
         \hline
    \end{tabular}
    \caption{A basic table with multiple rows and columns}
    \label{table:example01}
\end{table}
\begin{verbatim}
    \begin{table}[h!]
    \centering
    \begin{tabular}{ |c|c|c| } 
         \hline
         cell1 & cell2 & cell3 \\ 
         cell4 & cell5 & cell6 \\ 
         cell7 & cell8 & cell9 \\ 
         \hline
    \end{tabular}
    \caption{A basic table with multiple rows and columns}
    \label{table:example01}
\end{table}
\end{verbatim}

\section{Formatting commands}
Float barrier = A command from 'placeins' package.  Stops floats from extending beyond a certain point.  For ex:  Force figures to be placed within a given chapter.
\begin{verbatim}
    \FloatBarrier
\end{verbatim}

\section{Special characters}
\begin{verbatim}
    Degree symbol:  360$^{\circ}$
\end{verbatim}
Degree symbol:  360$^{\circ}$
\par
Italics and Bold:  This \textit{part} is in \textit{italics}.  This \textbf{part} is in \textbf{bold}.
%_-_-_-_-_-_-_-_-_-_-_-_-_-_-_-_-_
\section{In-Line Commenting}
To-Do items will show up in margins of printed document with pointers to todo flag
\begin{verbatim}
    \todo[inline]{Add in-line to-do item here}
    \todo{add margin to-do here}
\end{verbatim}
\todo[inline]{Add in-line to-do item here}
\todo{add margin to-do here}

Print list of to-do notes:
\listoftodos[Notes]
%_-_-_-_-_-_-_-_-_-_-_-_-_-_-_-_-_
\section{Chapters/Sections}
\subsection{This is a sub-sub-section}
\section*{No number here - just name}
This is a subsection with no section number
\begin{verbatim}
    \subsection*{subsection_title_here}
\end{verbatim}
\section{New Commands}
This command places *citation needed* in blue among the text
\begin{verbatim}
    \addcite
\end{verbatim}
\addcite


This command indicates an estimated value - useful to insert estimates in the text for later correction and citation
\begin{verbatim}
    \est{2 m}
\end{verbatim}
The elevation difference between the points was \est{2 m}

%--------------------------------------------------------------
%-------------------> END CONTENT HERE <-----------------------
%--------------------------------------------------------------
%% Don't try to compile this file, it is not a complete tex file

% \nobibliography{mendeley}  %% delete this line for final print

%% https://tex.stackexchange.com/questions/217517/handle-bib-in-subfiles
%% accessed 2017-12-17
\biblio
\end{document}